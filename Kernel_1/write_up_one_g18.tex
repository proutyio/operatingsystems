\documentclass[a4paper]{article}

%% Language and font encodings
\usepackage[english]{babel}
\usepackage[utf8x]{inputenc}
\usepackage[T1]{fontenc}

%% Sets page size and margins
\usepackage[a4paper,top=0.5in,bottom=0.25in,left=1in,right=1in,marginparwidth=1.75cm]{geometry}

%% Useful packages
\usepackage{amsmath}
\usepackage{graphicx}
\usepackage{subfig}
\usepackage[colorinlistoftodos]{todonotes}
\usepackage[colorlinks=true, allcolors=blue]{hyperref}
\usepackage{float}

% Support for typesetting algorithms
\usepackage{CJK}
\usepackage{algorithm}
\usepackage{algorithmicx}
\usepackage{algpseudocode}
\usepackage{amsmath}

\title{CS 444: Project One\\\Large{Getting Acquainted}}

\author{Kyle Prouty \&\\Nathaniel Whitlock}

\begin{document}
\maketitle

\section{Concurrency Solution}
Discuss concurrency solution at high level

\subsection{What do you think the main point of this assignment is?}

two points - learn how to get setup with linux kernel. Refresh understanding of threads and semaphores.(will add details later)

\subsection{How did you personally approach the problem? Design decisions, algorithms, etc.}
temp

\subsection{How did you ensure your solution was correct? Testing details?}
Temp

\subsection{What did you learn?}
Temp

\section{Command Log}
\subsection{Kernel assignment}
\begin{itemize}
  \item Clone git repo: 
    \begin{itemize}
      \item git clone git://git.yoctoproject.org/linux-yocto-3.19
    \end{itemize}
    \item Checkout v3.19.2 (From cloned repo): 
    \begin{itemize}
      \item git checkout tags/v3.19.2
    \end{itemize}
    \item Copy configuration file: 
    \begin{itemize}
      \item cp /scratch/files/config-3.19.2-yocto-standard linux-yocto-3.19/.config
    \end{itemize}
    \item Copy image and core files: 
    \begin{itemize}
      \item cp /scratch/files/bzImage-qemux86.bin /scratch/fall2017/18/
        \item cp /scratch/files/core-image-lsb-sdk-qemux86.ext4  /scratch/fall2017/18/
    \end{itemize}
    \item Source environment variables (BASH):
        \begin{itemize}
          \item source /scratch/files/environment-setup-i586-poky-linux
        \end{itemize}
        \item Kernel build: 
        \begin{itemize}
          \item make -j4 all
        \end{itemize}
        \item Run VM (From home): 
        \begin{itemize}
          \item qemu-system-i386 -gdb tcp::5518 -S -nographic -kernel bzImage-qemux86.bin -drive file=core-image-lsb-sdk-qemux86.ext4,if=virtio -enable-kvm -net none -usb -localtime --no-reboot --append ``root=/dev/vda rw console=ttyS0 debug''
         \end{itemize}
    \item Using GDB to continue VM:
    \begin{itemize}
      \item Establish secondary ssh connection to OSII server and login
        \item Launch gdb:
        \begin{itemize}
          \item target remote :5518
            \item continue
        \end{itemize}
    \item Logging into VM (From primary ssh connection):
    \begin{itemize}
      \item username: root
        \item password: (blank)
    \end{itemize}
    \end{itemize}
\end{itemize}


% qemu commands broken down
\subsection{Qemu-system-i386 Command Option Definitions}
\begin{itemize}
  \item -gdb: Wait for gdb connection, optionally pass argument for target host
    \item tcp: Specify the transmission control protocol port number
    \item -nographic: Disable graphical output and redirect serial I/Os to console
    \item -kernel: Specify the kernel image to use, use 'bzImage'
    \item -drive: Specify the file system, use file argument as a drive image
    \item if=virtio: Conditional check for virtual IO?
    \item -enable-kvm: Enable kernel virutal machine full virtualization support
    \item -net none: Enables zero network devices
    \item -usb: Enable the usb driver
    \item -localtime: 
    \item --no-reboot: Exit instead of rebooting
    \item --append ``root=/dev/vda rw console=ttyS0 debug'': Specify kernel command line
    \item -S: Freeze cpu at startup
\end{itemize}
\section{Version Control Log}
Look into creating a version control log from the GitHub repo

% Should contain
% - Kyle's work on the concurrency assignment
% - Nate's work on the write up, including kernel exercise solution

\section{Work Log}
Each week since the class started, both of us have met directly after Operating Systems in the Valley Public Library. During these sessions we worked on the following items:

%%% Document object examples %%%

%% Example of table environment
% \begin{center}
% \begin{tabular}{ |l|c| }
%  \hline
%  \textbf{Action} & \textbf{Savings} \\
%  \hline
%  Reduce driving by 10\% & 0.50T \\
%  \hline
%  Give up second vehicle & 0.50T  \\ 
%  \hline
%  Compost organic kitchen wastes & 0.2T \\
%  \hline
%  Stop bagging grass cuttings & 0.2T \\
%  \hline
%  Perform regular appliance maintenance & 0.2T \\
%  \hline
%  Turn off water when brushing teeth & 0.10T \\
%  \hline
%  Use dish washer's no heat drying & 0.10T \\
%  \hline
%  Reduce paper usage & 0.10T \\
%  \hline
%  Remove roof racks & 0.10T \\
%  \hline
%  \textbf{Total} & \textbf{2.0T}\\
%  \hline
% \end{tabular}
% \end{center}

% \begin{center}
% Table 1. List of actions to sum to 2.0 tons of GHG savings
% \end{center}

%% Example of embedding image
% \vspace{0.1in}
% \begin{center}
%     \centering
%     \includegraphics[width=0.65\linewidth]{Eco_Footprint.PNG}
%     \\Fig 1. Image of Ecological Footprint result
% \end{center}




\end{document}
